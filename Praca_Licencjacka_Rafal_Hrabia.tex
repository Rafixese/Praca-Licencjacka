\documentclass[a4paper,12pt,oneside]{book} % nie: report!


% pakiety
\usepackage{polski} % lepiej to zamiast babel!
\usepackage[utf8]{inputenc} % w razie kłopotów spróbować: \usepackage[utf8x]{inputenc}
\usepackage{fancyhdr} % nagłówki i stopki
\usepackage{indentfirst} % WAŻNE, MA BYĆ!
\usepackage[pdftex]{graphicx} % to do wstawiania rysunków
\usepackage{amsmath} % to do dodatkowych symboli, przydatne
\usepackage[pdftex,
            left=1in,right=1in,
            top=1in,bottom=1in]{geometry} % marginsy
\usepackage{amssymb} % to też do dodatkowych symboli, też przydatne
\usepackage{pdfpages}
\usepackage{lipsum}
\usepackage{multirow}
\usepackage{listings}
\usepackage{caption}
\DeclareCaptionType{code}[Listing][Spis listingów] 

% definicje nagłówków i stopek
\pagestyle{fancy}
\renewcommand{\chaptermark}[1]{\markboth{#1}{}}
\renewcommand{\sectionmark}[1]{\markright{\thesection\ #1}}
\fancyhf{}
\fancyhead[LE,RO]{\footnotesize\bfseries\thepage}
\fancyhead[LO]{\footnotesize\rightmark}
\fancyhead[RE]{\footnotesize\leftmark}
\renewcommand{\headrulewidth}{0.5pt}
\renewcommand{\footrulewidth}{0pt}
\addtolength{\headheight}{1.5pt}
\fancypagestyle{plain}{\fancyhead{}\cfoot{\footnotesize\bfseries\thepage}\renewcommand{\headrulewidth}{0pt}}


% interlinia
\linespread{1.25}


% treść
\begin{document}
\sloppy
\thispagestyle{empty}
\includepdf{stronatytulowa}
\newpage{}

\thispagestyle{empty}
\newpage{}

\tableofcontents{}

\chapter*{Wstęp}
\addcontentsline{toc}{chapter}{Wstęp}
\label{Wstep}
%Mega ogólne sprawy dotyczące tego co chce zrobić
Sieci neuronowe mają za zadanie naśladowanie zachowań sieci neuronów znajdujących się w mózgu człowieka. Zostały stworzone do rozwiązywania zadań trudnych lub prawie niemożliwych do opisania za pomocą reguł, wyrażeń logicznych i innych narzędzi programistycznych. Wraz z rozwojem sieci neuronowych powstało wiele wariantów, które ze względu na swoją budowę lepiej lub gorzej sprawdzają się w różnych problemach. W przypadku rozpoznawania obrazów w postaci dwu-wymiarowej macierzy dla danych monochromatycznych lub trój-wymiarowej macierzy dla zdjęć kolorowych jednym z najlepszych wyborów będą sieci konwolucyjne. Sieci te rozpoznają wzorce, poczynając od linii horyzontalnych i wertykalnych, a w dalszych warstwach kończąc na skomplikowanych strukturach. Budowa i działanie sieci konwolucyjnych daje wielki potencjał do klasyfikacji obrazów. Ta praca przedstawi przykład takiej klasyfikacji wieloklasowej z użyciem sieci konwolucyjnych na przykładzie alfabetu Amerykańskiego Języka Migowego. 

\chapter*{Cel i zakres pracy}
\addcontentsline{toc}{chapter}{Cel i zakres pracy}
\label{Cel i zakres pracy}
%celem pracy jest napisanie programu bla bla
Celem pracy jest stworzenie programu wyposażonego w wytrenowany model do rozpoznawania obrazu, który w czasie rzeczywistym używając kamery internetowej będzie w stanie odczytać i wyświetlić na ekranie transkrypcję znaków języka migowego pokazywanych przez osobę znajdującą się w polu widzenia kamery. Dodatkowo do pracy będą składać się: utworzenie zbioru danych składającego się z około 50000 zdjęć zawierających wszystkie litery alfabetu ASL, utworzenie modelu z warstw konwolucyjnych i gęstych oraz wytrenowanie modelu i tuning parametrów. W części teoretycznej pracy zostanie przybliżony temat języka migowego American Sign Language, jak również temat konwolucyjnych sieci neuronowych i sposobu działania modeli.

\chapter{Język migowy American Sign Language}
American Sign Language (Amerykański Język Migowy, ASL) to złożony język wizualno-przestrzenny używany przez ludzi głuchoniemych w Stanach Zjednoczonych Ameryki oraz anglojęzycznych częściach Kanady. Jest to w pełni kompletny język naturalny. ASL jest językiem natywnym dla wielu głuchoniemych mężczyzn, kobiet i dzieci, a także niektórych słyszących dzieci w rodzinach, gdzie opiekunowie prawni są niesłyszący.\cite{nakamura}
\section{Historia}
Pochodzenie dzisiejszej społeczności osób głuchoniemych w Stanach Zjednoczonych jest powszechnie utożsamiane z założeniem pierwszej szkoły dla niesłyszących - American School for the Deaf (ASD), założonej w 1817 roku w Hartford, Connecticut. Przed założeniem szkoły ASD na terenie USA działało wiele niezależnych społeczności, poczynając od małych grup o wielkości pojedynczej rodziny do większych - całych wsi. W małych społecznościach uformowały się niezależne znaki i systemy języka migowego, które są obecne po dziś dzień w tych środowiskach. Istnieją dowody na to, że głuchonieme dzieci kształtowały swoje własne systemy języków migowych, które były o wiele bardziej wyrafinowane od tych, używanych w społeczności, w których się znajdowały.\cite{bahan}

Istnieją również doniesienia o innym niezależnie uformowanym systemie języka migowego Martha’s Vineyard Sign Language (MVSL), który istniał przed założeniem American School for the Deaf. Język ten był głównie używany w wsi Chilmark na wybrzeżu Massachusetts. Powstanie MVSL zapoczątkował fakt, że wspomniana społeczność Chilmark miała wysoki odsetek mutacji genetycznych prowadzących do głuchoty. W skali miasteczka około 4 procent mieszkańców było głuchoniemych. Wynikało to z wysokiego odsetka mieszanych małżeństw od wielu pokoleń, począwszy od hrabstwa Kent w Angli, zanim wieś Chilmark została założona w roku 1690.\cite{groce} Mieszkańcy, którzy nie byli głuchoniemi również posługiwali się językiem MVSL, wówczas gdy znajdowali się w towarzystwie osób z niepełnosprawnością ale również, gdy w gronie rozmówców nie było osoby głuchoniemej. Język był używany do czasu założenia szkoły dla osób głuchoniemych ASD w Hartford. Dzieci z wsi Chilmark zaczęto wysyłać do szkoły American School for the Deaf we wczesnych latach dwudziestych XIX wieku. Skutkowało to zatarciem się języków MVSL i nowego języka migowego ewoluującego w dzisiejszy język ASL.\cite{bahan}

Kiedy szkoła dla głuchoniemych została założona, stała się miejscem, gdzie wiele pomniejszych systemów migowych stykało i mieszało się ze sobą przez 175 lat. Z mieszanki tych języków powstał dzisiejszy język ASL. Głuchoniemy nauczyciel - Laurent Clerc, pochodzący z Francji, był pierwszym nauczycielem w wspomnianej placówce. Z kraju swojego pochodzenia przywiózł wiedzę o Francuskim Języku Migowym, którego znaków nauczał w amerykańskiej szkole. Ta sytuacja spowodowała bardzo mocne doprawienie wówczas powstającego języka ASL o aspekty zaciągnięte z Francuskiego Języka Migowego. Na dzień dzisiejszy około 60 procent współczesnego systemu ASL opiera się na starym migowym języku Francuskim.\cite{bahan}


\section{Populacja}

\section{Dystrybucja geograficzna}
\section{Składnia języka}


\chapter{Konwolucyjne sieci neuronowe i rozpoznawanie obrazów}

\section{Sekcja C}
\section{Sekcja D}

\chapter{Implementacja aplikacji do rozpoznawania języka migowego ASL}

\chapter{Podsumowanie}


\listoftables{} % jeśli są tabele
\addcontentsline{toc}{chapter}{Spis tabel}

\listoffigures{} % jeśli są tabele
\addcontentsline{toc}{chapter}{Spis rysunków}

\listofcodes
\addcontentsline{toc}{chapter}{Spis listingów}

\addcontentsline{toc}{chapter}{Bibliografia}
\bibliographystyle{ieeetr}
\bibliography{bibliography/cite}


\end{document}
