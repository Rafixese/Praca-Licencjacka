\documentclass[a4paper,12pt]{book} % nie: report!


% pakiety
\usepackage{polski} % lepiej to zamiast babel!
\usepackage[utf8]{inputenc} % w razie kłopotów spróbować: \usepackage[utf8x]{inputenc}
\usepackage{fancyhdr} % nagłówki i stopki
\usepackage{indentfirst} % WAŻNE, MA BYĆ!
\usepackage[pdftex]{graphicx} % to do wstawiania rysunków
\usepackage{amsmath} % to do dodatkowych symboli, przydatne
\usepackage[pdftex,
            left=1in,right=1in,
            top=1in,bottom=1in]{geometry} % marginsy
\usepackage{amssymb} % to też do dodatkowych symboli, też przydatne
\usepackage{pdfpages}
\usepackage{lipsum}


% definicje nagłówków i stopek
\pagestyle{fancy}
\renewcommand{\chaptermark}[1]{\markboth{#1}{}}
\renewcommand{\sectionmark}[1]{\markright{\thesection\ #1}}
\fancyhf{}
\fancyhead[LE,RO]{\footnotesize\bfseries\thepage}
\fancyhead[LO]{\footnotesize\rightmark}
\fancyhead[RE]{\footnotesize\leftmark}
\renewcommand{\headrulewidth}{0.5pt}
\renewcommand{\footrulewidth}{0pt}
\addtolength{\headheight}{1.5pt}
\fancypagestyle{plain}{\fancyhead{}\cfoot{\footnotesize\bfseries\thepage}\renewcommand{\headrulewidth}{0pt}}


% interlinia
\linespread{1.25}


% treść
\begin{document}
\sloppy



\thispagestyle{empty}

\includepdf{stronatytulowa}
% najpierw uzupełnij w 'stronatytulowa.odt' openoffice i wyeksportuj do 'stronatytulowa.pdf'

\newpage{}

\thispagestyle{empty}

\newpage{}



\tableofcontents{}

\chapter{}
\section{Wstęp}
\label{Wstep}
%Mega ogólne sprawy dotyczące tego co chce zrobić
\lipsum[1]

\section{Cel i teza pracy}
\label{Cel i teza pracy}
%celem pracy jest napisanie programu bla bla
\lipsum[1]

\section{Zakres pracy}
\label{Zakres pracy}
%Trza zrobic research bla bla
\begin{itemize}
	\item Przegląd metod analizy obrazu;
	\item Przegląd dostępnych sieci neuronowych dedykowanych do analizy obrazu;
	\item Opracowanie specyfikacji sprzętowej niezbędnej do realizacji prototypowego systemu;
	\item Wytworzenie oprogramowania realizującego wykrywanie łuku elektrycznego;
	\item Analiza wyników i szybkości działania zastosowanych metod YoLo, R-CNN;
\end{itemize}
\lipsum[1]

\chapter{Rozdział o czymś tam}

\section{Sekcja A}


\section{Sekcja B}


\chapter{Rozdział o czymś innym}

\section{Sekcja C}


\section{Sekcja D}


\listoftables{} % jeśli są tabele
\addcontentsline{toc}{chapter}{Spis tabel}

\listoffigures{} % jeśli są tabele
\addcontentsline{toc}{chapter}{Spis rysunków}

\addcontentsline{toc}{chapter}{Bibliografia}
\bibliographystyle{ieeetr}
\bibliography{bibliography/yolo}


\end{document}
